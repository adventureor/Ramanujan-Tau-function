\documentclass{article}
\usepackage{amssymb,amsmath,mathtools}
\usepackage{amsfonts,booktabs,mathrsfs}
\usepackage{amsthm}
\usepackage[dvipsnames]{xcolor}
\usepackage{hyperref}
\hypersetup{
	colorlinks=true,
	linkcolor=cyan,
	filecolor=blue,      
	urlcolor=red,
	citecolor=green,
}
\newtheorem{theorem}{Theorem}
\newtheorem{definition}[theorem]{Definition}
\newtheorem{example}[theorem]{Example}
\newtheorem{corollary}[theorem]{Corollary}
\newtheorem{lemma}[theorem]{Lemma}
\newtheorem{proposition}[theorem]{Proposition}
\newcommand{\Mod}[1]{\ (\mathrm{mod}\ #1)}
\begin{document}
\title{Ramanujan $\tau$-Functions}
\author{Yu Xiong}
\maketitle
\section{First Note}
\begin{definition}
    For any $A \in GL(2,\mathbb{R})$, the \textcolor{blue}{slash operator} defined on $f:\mathcal{H}\to \mathbb{C}$ is
    \[
        f|_kA(z)=(\text{det}(A))^{k/2}j_A(z)^{-k}f(Az),
    \] we have $j_A(z)=cz+d$, if $A=\begin{pmatrix}
        a&b\\
        c&d
    \end{pmatrix}$.
\end{definition}
\begin{definition}
    \[\Delta_n:=\left\{\begin{pmatrix}
        a&b\\
        0&d
    \end{pmatrix}\big| ad=n, 0\leq b < d\right\}.\]
\end{definition}
\begin{lemma}
    There is a one to one correspondence \[\Delta_n\times SL(2,\mathbb{Z})\leftrightarrow  SL(2,\mathbb{Z}) \times \Delta_n.\] That is for any $\rho \in \Delta_n$, $\tau \in SL(2,\mathbb{Z})$, there exist unique $\tau'\in \Gamma,\rho' \in \Delta_n$, such that $\rho \tau =\tau' \rho'$.
\end{lemma}
\begin{proof}
    If $\rho=\begin{pmatrix}
        a&b\\
        0&d
    \end{pmatrix} \in \Delta_n$ and $\tau=\begin{pmatrix}
        \alpha&*\\
        \gamma&\delta
    \end{pmatrix} \in SL(2,\mathbb{Z})$, we want to find $\tau'= \begin{pmatrix}
        \alpha'&*\\
        \gamma'&\delta'
    \end{pmatrix} \in SL(2,\mathbb{Z})$ and $\rho'\begin{pmatrix}
        a'&b'\\
        0&d'
    \end{pmatrix} \in \Delta_n$ such that $\rho \tau =\tau' \rho'$. This is we want to solve:\[
    \begin{pmatrix}
        a&b\\
        0&d
    \end{pmatrix}
    \begin{pmatrix}
        \alpha&*\\
        \gamma&\delta
    \end{pmatrix}=
    \begin{pmatrix}
        \alpha'&*\\
        \gamma'&\delta'
    \end{pmatrix}
    \begin{pmatrix}
        a'&b'\\
        0&d'
    \end{pmatrix}.
    \]
    This is \[
    \begin{pmatrix}
        a\alpha+b\gamma&*\\
        d\gamma&d\delta
    \end{pmatrix}=
    \begin{pmatrix}
        a'\alpha'&*\\
        a'\gamma'&b'\gamma'+d'\delta'
    \end{pmatrix},
    \]
    and this is equal to
    \begin{equation*}
        \begin{cases}
            a'\alpha'&=a\alpha+b\gamma;\\
            a'\gamma'&=d\gamma;\\
            b'\gamma'+d'\delta'&=d\delta.
        \end{cases}
    \end{equation*}
    For $\gamma\neq 0$, we can see that $a'\mid (a\alpha+b\gamma)$ and $a' \mid d\gamma$, so \[(\frac{a\alpha+b\gamma}{a'},\frac{d\gamma}{a'})=(\alpha',\gamma').\] From $\alpha'\delta'-\beta'\gamma'=1$, we can see that $(\alpha',\gamma')=1$, so $a'=(a\alpha+b\gamma,d\gamma)$. In addition, we have \begin{equation}\label{eqn 1}
        \begin{cases}
            a'&=(a\alpha+b\gamma,d\gamma);\\
            d'&=n/(a\alpha+b\gamma,d\gamma);\\
            \alpha'&=(a\alpha+b\gamma)/(a\alpha+b\gamma,d\gamma);\\
            \gamma'&=d\gamma/(a\alpha+b\gamma,d\gamma).
        \end{cases}
    \end{equation}
    % We have $\alpha'\delta' \equiv 1 \Mod{|\gamma'|}$, so $\delta'\equiv x \Mod{|\gamma'|}$, for some $x\in\mathbb{Z}$, and we have $\delta'=x+n\gamma'$. We can see that for any $n\in \mathbb{Z}$, we can make $\delta'\alpha'-*\gamma'=1$. So we have $\gamma'b'+d'(x+\gamma' n)=\delta d$, and that is \begin{equation}\label{1}
    % \gamma'(b'+nd')=\delta d-xd'.
    % \end{equation}
    % We can see that $\gamma'\neq 0$, if $\gamma\neq 0$, and if we modulo $|\gamma'|$ for RHS of equation \ref{1}, we will have \[\delta d-xd'\equiv \delta d- d'/\alpha'\equiv \frac{n\alpha\delta+b\gamma\delta d-n}{a\alpha+b\gamma}=0\Mod{|\gamma'|}.\]Hence, we have $b'=\frac{\delta d-xd'}{\gamma'}-nd'$ and taking a suitable $n$, we get the $b'$.\\
    % If $\gamma=0$, then we have 
    % \[\begin{pmatrix}
    %     a&b\\
    %     0&d
    % \end{pmatrix}\begin{pmatrix}
    %     1&u\\
    %     0&1
    % \end{pmatrix}=
    % \begin{pmatrix}
    %     1&v\\
    %     0&1
    % \end{pmatrix}
    % \begin{pmatrix}
    %     a&b'\\
    %     0&d
    % \end{pmatrix},\]
    % so we take $b'\equiv au+b\Mod{d}$ and $v=d^{-1}(au+b-dv)$.\\
    % For surjection, we need some matrices transform.
    We can see that $a',\alpha',\gamma'$ are positive integers. For $d'$, we have \[(a,\gamma)=(a\alpha,\gamma)=(a\alpha+b\gamma,\gamma),\]is a divisor of $a$ due to $(\alpha,\gamma)=1$, so we have $(a\alpha+b\gamma,d\gamma)\mid ad$ and $d'$ is an integer. 

    We have $\alpha'\delta'\equiv 1\Mod{|\gamma'|}$ and $(\alpha',\gamma')=1$, there is a $\delta''\in\mathbb{Z}$, such that $\delta'\equiv \delta''\Mod{|\gamma'|}$, so $\delta'=\delta''+\gamma' m$. From $b'\gamma'+d'\delta'=d\gamma$, we have $(b'+md')\gamma'+\delta''d'=d\delta$. 
    \begin{equation*}
        \begin{aligned}
            \alpha'd\delta-\alpha'\delta''d'&\equiv \alpha'd\delta-d' &\Mod{|\gamma'|}\\
            &\equiv \frac{ad\alpha\delta+bd\delta\gamma-n}{(a\alpha+b\gamma,d\gamma)} &\Mod{|\gamma'|}\\
            &\equiv \frac{(\alpha\delta-1)n}{(a\alpha+b\gamma,d\gamma)}+\gamma'\delta b &\Mod{|\gamma'|}\\
            &\equiv 0&\Mod{|\gamma'|}.
        \end{aligned}
    \end{equation*}
    So we find $b'$ and $\delta'$.

    If $\gamma=0$, then we have 
    \[\begin{pmatrix}
        a&b\\
        0&d
    \end{pmatrix}\begin{pmatrix}
        1&u\\
        0&1
    \end{pmatrix}=
    \begin{pmatrix}
        1&v\\
        0&1
    \end{pmatrix}
    \begin{pmatrix}
        a&b'\\
        0&d
    \end{pmatrix},\]
    so we take $b'\equiv au+b\Mod{d}$ and $v=d^{-1}(au+b-dv)$, and $v$ is an integer for $d\mid (au+b-dv)$.

    For surjection, we need some matrices transform. If we have $\tau'\in \Gamma$, $\rho'\in \Delta_n$, we want to find $\tau$ and $\gamma$. We can assume that $\tau'\rho'=\begin{pmatrix}
        x&y\\
        z&w
    \end{pmatrix}$, where $xw-yz=n$. Then we have 
    \[
    \begin{pmatrix}
        x&y\\
        z&w
    \end{pmatrix}
    \begin{pmatrix}
        w/(z,w)&*\\
        -z/(z,w)&*
    \end{pmatrix}=
    \begin{pmatrix}
        *&*\\
        0&*
    \end{pmatrix}.
    \]With a matrix like $\begin{pmatrix}
        1&n\\
        0&1
    \end{pmatrix}$ we will get the surjection.
\end{proof}
\section{Second Note}
\begin{definition}\label{def 1}
    Let $\chi:\mathbb{Z}\to\mathbb{C}$, and $\chi(ab)=\chi(a)\chi(b)$, for any $(a,b)\in \mathbb{Z}^2$. If $n\geq 1$, we define \textcolor{blue}{Hecke operators} $T_n$ on $f:\mathcal{H}\to \mathbb{C}$ to be
    \[T_nf:=\frac{1}{n}\sum_{ad=n}\chi(a)a^k\sum_{b\Mod{d}}f(\frac{a\tau+b}{d}).\]
\end{definition}
\begin{theorem}
    For any $m,n\geq 1$, we have \[T_mT_nf=\sum_{d\mid (m,n)}\chi(d)d^{k-1}T_{\frac{mn}{d^2}}f\]
\end{theorem}
\begin{proof}
    From Definition \ref{def 1}, we have 
    \begin{equation}
        \begin{aligned}
            &mnT_mT_nf(\tau)\\
            &=\sum_{\substack{a_1d_1=m\\a_2d_2=n}}\chi(a_1a_2)(a_1a_2)^k \sum_{\substack{b_1\Mod{d_1}\\b_2\Mod{d_2}}}f\left(\begin{pmatrix}
            a_1&b_1\\
            0&d_1
        \end{pmatrix}
        \begin{pmatrix}
            a_2&b_2\\
            0&d_2
        \end{pmatrix}\tau\right)\\
        &=\sum_{d\mid (m,n)}\sum_{\substack{a_1d_1=m\\a_2d_2=n\\(a_1,d_2)=d}}\chi(a_1a_2)(a_1a_2)^k \sum_{\substack{b_1\Mod{d_1}\\b_2\Mod{d_2}}}f\left(\begin{pmatrix}
            a_1&b_1\\
            0&d_1
        \end{pmatrix}
        \begin{pmatrix}
            a_2&b_2\\
            0&d_2
        \end{pmatrix}\tau\right)\\
        =&\sum_{d\mid (m,n)}\chi(d)d^k\sum_{\substack{a_1'd_1=m/d\\a_2d_2'=n/d\\(a_1',d_2')=1}}\chi(a_1'a_2)(a_1'a_2)^k \sum_{\substack{b_1\Mod{d_1}\\b_2\Mod{d d_2'}}}f\left(\begin{pmatrix}
            da_1'a_2&d(a_1'b_2+b_1d_2')\\
            0&dd_1d_2'
        \end{pmatrix}\tau\right).
        \end{aligned}
    \end{equation}
    We can see that for any $a\in \mathbb{R}$ and $\gamma\in SL(2,\mathbb{Z})$, we have $f(a\gamma\tau)=f(\gamma\tau)$, so \[f\left(\begin{pmatrix}
            da_1'a_2&d(a_1'b_2+b_1d_2')\\
            0&dd_1d_2'
        \end{pmatrix}\tau\right)=f\left(\begin{pmatrix}
            a_1'a_2&a_1'b_2+b_1d_2'\\
            0&d_1d_2'
        \end{pmatrix}\tau\right).\]

    I claim that for fixed $a_1',a_2,d_1,d_2'$ and $(a_1',d_2')=1$, if $b_1$ runs over $\Mod{d_1}$ and $b_2$ runs over $\Mod{d_2'}$, then we have $a_1'b_2+b_1d_2'$ runs over $\Mod{d_1d_2'}$ one times. 

    $(b_1,b_2)$ has $d_1d_2'$ ways to take, so we just need to show that taking different $(b_1^*,b_2^*)$ we have $a_1'b_2+b_1d_2'\not\equiv a_1'b_2^*+b_1^*d_2'\Mod{d_1d_2'}$. In fact, if they are equal, we have $a_1'(b_2-b_2^*)\equiv d_2'(b_1^*-b_1) \Mod{d_1d_2'}$, so we have $d_2'\mid a_1'(b_2-b_2^*)$. But $(a_1',d_2')=1$, so $b_2\equiv b_2^*\Mod{d_2'}$. So we prove the claim.

    So we have \begin{equation*}
        \sum_{\substack{b_1\Mod{d_1}\\b_2\Mod{d d_2'}}}f\left(\begin{pmatrix}
            a_1'a_2&a_1'b_2+b_1d_2'\\
            0&d_1d_2'
        \end{pmatrix}\tau\right)=d\sum_{b \Mod{d_1d_2'}}f\left(\begin{pmatrix}
            a_1'a_2&b\\
            0&d_1d_2'
        \end{pmatrix}\tau\right)
    \end{equation*}
    We have
    \begin{equation*}
        \begin{aligned}
        &mnT_mT_n\\
        &=\sum_{d\mid (m,n)}\chi(d)d^{k+1}\sum_{\substack{a_1'd_1=m/d\\a_2d_2'=n/d\\(a_1',d_2')=1}}\chi(a_1'a_2)(a_1'a_2)^k
        \sum_{b \Mod{d_1d_2'}}f\left(\begin{pmatrix}
            a_1'a_2&b\\
            0&d_1d_2'
        \end{pmatrix}\tau\right)\\
        &=\sum_{d\mid (m,n)}\chi(d)d^{k+1}\sum_{ad=\frac{mn}{d^2}}\chi(a)a^k\sum_{b\Mod{d}}f\left(\begin{pmatrix}
            a&b\\
            0&d
        \end{pmatrix}\tau\right)\\
        &=\sum_{d\mid(m,n)}\chi(d)d^{k+1}T_{\frac{mn}{d^2}}f.
        \end{aligned}
    \end{equation*}
\end{proof}
Recall that we have 
\begin{definition}
    \[\Delta_n:=\left\{\begin{pmatrix}
        a&b\\
        0&d
    \end{pmatrix}\big| ad=n, 0\leq b < d\right\}.\]
\end{definition}
\begin{lemma}\label{lem 4}
    There is a one to one correspondence \[\Delta_n\times SL(2,\mathbb{Z})\leftrightarrow  SL(2,\mathbb{Z}) \times \Delta_n.\] That is for any $\rho \in \Delta_n$, $\tau \in SL(2,\mathbb{Z})$, there exist unique $\tau'\in \Gamma,\rho' \in \Delta_n$, such that $\rho \tau =\tau' \rho'$.
\end{lemma}
If we take $\chi(a)=1$ for all $a\in \mathbb{Z}$, we have 
\begin{equation*}
    \begin{aligned}
        T_nf&=\frac{1}{n}\sum_{ad=n}\chi(a)a^k\sum_{b\Mod{d}}f(\frac{a\tau+b}{d})\\
        &=n^{k-1}\sum_{\rho \in \Delta_n}d^{-k}f(\rho \tau)\text{\qquad \qquad for ad=n}\\
        &=n^{k/2-1}\sum_{\rho \in \Delta_n}f|_k\rho (\tau).
    \end{aligned}
\end{equation*}
So we have 
\begin{proposition}
    $T_n:M_k(\Gamma)\to M_k(\Gamma)$, and if \[f=\sum_{m=0}^{\infty}a(m)e(mz),\] then we have $T_nf(z)=\sum_{m=1}^{\infty}a_n(m)e(mz)$, where\[a_n(m)=\sum_{d\mid (m,n)}d^{k-1}a(mnd^{-2}).\]
\end{proposition}
\begin{proof}
    From Lemma \ref{lem 4}, we have 
    \begin{equation*}
        \begin{aligned}
            T_nf|_k\gamma&=n^{k/2-1}\sum_{\rho\in \Delta_n}f|_k\rho\gamma(\tau)\\
            &=n^{k/2-1}\sum_{\rho'\in \Delta_n}f|_k\gamma'\rho'(\tau)\\
            &=T_nf.
        \end{aligned}
    \end{equation*}
    For the Fourier expansion, we have 
    \begin{equation*}
        \begin{aligned}
            T_nf(z)&=\frac{1}{n}\sum_{ad=n}a^k\sum_{0\leq b <d}f(\frac{az+b}{d})\\
            &=\frac{1}{n}\sum_{ad=n}a^k\sum_{0\leq b <d}\sum_{m=0}^{\infty}a(m)e(\frac{amz+bm}{d})\\
            &=\frac{1}{n}\sum_{ad=n}a^k\sum_{m=0}^{\infty}a(m)\sum_{0\leq b <d}e(\frac{amz+bm}{d})\\
            &=\frac{1}{n}\sum_{ad=n}a^kd\sum_{m=0}^{\infty}a(m)e(amz)\\
            &=\sum_{N=0}^{\infty}\sum_{\substack{ad=n\\am=N}}a^{k-1}a(md)e(Nz).
        \end{aligned}
    \end{equation*}
\end{proof}
\begin{corollary}
    We have $\Delta(\tau)=\sum_{n=1}^{\infty}\tau(n)q^n$, and \[\tau(n)\tau(m)=\sum_{d\mid (m,n)}d^{11}\tau(\frac{mn}{d^2}).\]
\end{corollary}
\section{Third Note}
In this part, I will introduce the following theorem, whose proof is from \href{https://math.stackexchange.com/q/84426}{Math Stackexchange}, and there are some corollaries giving some congruence of $\tau$ functions, which is from \cite{MR21951}.
\begin{theorem}\label{Ramanathan}
    \[(1-n)\tau(n)=24\sum_{j=1}^{n-1}\sigma(j)\tau(n-j).\]
\end{theorem}
\begin{definition}
    $\delta_k$ is an operator from holomorphic functions to holomorphic functions. \[\delta_k(f):=12\theta (f)-kE_2f,\] where $\theta(f):=q\frac{df}{dq}$, $q=e^{2\pi i \tau}$ and \[E_2(\tau)=1-2\sum_{n=1}^{\infty}\sigma q^n.\]
\end{definition}
\begin{lemma}
    $\delta_k: M_k(\Gamma)\to M_{k+2}(\Gamma)$ and $\delta_k(S_k(\Gamma))\subset{S_{k+2}(\Gamma)}$.
\end{lemma}
\begin{proof}
    To prove $\delta_k(M_k(\Gamma))\subset M_{k+2}(\Gamma)$, we just need to show for any $f\in M_k(\Gamma)$ and $\alpha\in \Gamma$, we have 
    \begin{equation}\label{*}
        (\delta_kf)|_{k+2}\alpha=\delta_kf.
    \end{equation}
    We can see that $(\delta_kf)|_{k+2}\alpha=12(\theta_kf)|_{k+2}\alpha-kE_2|_2\alpha f|_k\alpha$. Firstly, we have $\theta (f)=q\frac{df}{dq}=\frac{1}{2\pi i}f'(\tau)$, so we have 
    \[\theta (f)|_{k+2}\alpha=\frac{1}{2\pi i}f'|_{k+2}\alpha=\frac{1}{2\pi i}(c\tau+d)^{-k-2}f'(\alpha\tau).\]
    We have $f(\alpha\tau)=(c\tau+d)^{k}f(\tau)$, and taking the derivative of the equation we have $\frac{1}{(c\tau+d)^2}f'(\alpha\tau)=kc(c\tau+d)^{k-1}f(\tau)+(c\tau+d)^kf'(\tau)$. So we have 
    \[f'(\alpha\tau)=kc(c\tau+d)^{k+1}f(\tau)+(c\tau+d)^{k+2}f'(\tau).\]
    Taking the $f'(\alpha\tau)$ to $(\theta_kf)|_{k+2}\alpha$, we have 
    \begin{equation*}
        \begin{aligned}
            (\theta_kf)|_{k+2}\alpha&=\frac{1}{2\pi i}(c\tau+d)^{-k-2}f'(\alpha\tau)\\
            &=\frac{1}{2\pi i}\left(\frac{kc}{c\tau+d}f(\tau)+f'(\tau)\right)\\
            &=\frac{kcf(\tau)}{2\pi i(c\tau+d)}+\theta f.
        \end{aligned}
    \end{equation*}
    So we have 
    \begin{equation*}
        \begin{aligned}
            (\delta_kf)|_{k+2}\alpha-\delta_kf&=12(\theta_kf)|_{k+2}\alpha-kE_2|_2\alpha f-\left(12\theta (f)-kE_2f\right)\\
            &=kf(\tau)\left(\frac{12c}{2\pi i(c\tau+d)}-E_2|_2\alpha+E_2\right)\\
            &=0.
        \end{aligned}
    \end{equation*}
    Last equation is from $E_2|_2\alpha=E_2+\frac{12c}{2\pi i(c\tau+d)}$.
\end{proof}
\begin{proof}[Proof of Theorem \ref{Ramanathan}]
    We have $\delta_k\Delta\in S_{14}(\Gamma)=0$, so $\delta_k\Delta=0$. So we have 
    \[\delta_k\Delta=12\sum_{n=1}^{\infty}n\tau(n)q^n-12E_2\Delta=0.\]
    Hence, we have
    \begin{equation*}
        \begin{aligned}
            \sum_{n=1}^{\infty}n\tau(n)q^n&=\left(1-24\sum_{n=1}^{\infty}\sigma(n)q^n\right)\left(\sum_{n=1}^{\infty}\tau(n)q^n\right)\\
            &=\sum_{n=1}^{\infty}\tau(n)q^n-24\sum_{n=1}^{\infty}\left(\sum_{m=1}^{n-1}\tau(m)\sigma(n-m)q^n\right).
        \end{aligned}
    \end{equation*}
\end{proof}
\begin{corollary}
    We have 

    (a)$\tau(2n)\equiv 0\Mod{8}$.

    (b)$\tau(3n-1)\equiv \tau(3n) \equiv 0 \Mod{3}$.

    (c)$\tau(4n-1)\equiv 0 \Mod{4}$.
\end{corollary}
\begin{proof}
    We know that $\tau(n)$ are integers, so if we modulo $24$ on the two sides of the equation of Theorem \ref{Ramanathan}, we have 
    \[(1-n)\tau(n)\equiv 0\Mod{24}.\]
    If $n=2k$, we have $(1-2k)\tau(2k)\equiv 0\Mod{8}$, so we have (a).\\
    If $n=3k-1$, we have $(2-3k)\tau(3k-1)\equiv 2\tau(n) \equiv 0\Mod{3}$, so we get (b).\\
    If $n=4k-1$, we have $(2-4k)\tau(4k-1)\equiv 2(1-2k)\tau(4k-1) \equiv 0\Mod{24}$, so we have $(1-2k)\tau(4k-1)\equiv 0 \Mod{4}$, and we have (c).
\end{proof}

\begin{proposition}
    $\tau(n)$ is odd if and only if $n$ is the square of an odd number
\end{proposition}
This proposition is proved by Hansraj Gupta, but I haven't check.
\begin{proposition}
    We have 

    (i)$\tau(6n-1)\equiv 0 \Mod{6}$.
    
    (ii)$\tau(8n-1)\equiv 0 \Mod{8}$.
\end{proposition}
\begin{proof}
    For (i), we know that $\left(\frac{-1}{6}\right)=-1$, so from the corollary and proposition, we have (i).

    For (ii), we need the observation that $\tau(8n-r)\sigma(r-1)\equiv 0\Mod{2}$, for $1<r\leq 8n-1$. From the proposition, we can see that $\tau(8n-r)$ is odd if and only if $r\equiv 7\Mod{8}$. Then we can see that $r-1\equiv 6 \Mod{8}$, so we can see that there is a prime factor of $r-1$ has odd power, and from the multiplication of $\sigma$, we get $\sigma(r-1)\equiv 0 \Mod{2}$. Hence, we get (ii).
\end{proof}




\bibliographystyle{plain}
% \bibliographystyle{gbt7714-numerical}
\bibliography{ref.bib}
\end{document}