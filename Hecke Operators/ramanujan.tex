\documentclass{article}
\usepackage{amssymb,amsmath,mathtools}
\usepackage{amsfonts,booktabs,mathrsfs}
\usepackage{amsthm}
\usepackage[dvipsnames]{xcolor}
\newtheorem{theorem}{Theorem}
\newtheorem{definition}[theorem]{Definition}
\newtheorem{example}[theorem]{Example}
\newtheorem{corollary}[theorem]{Corollary}
\newtheorem{lemma}[theorem]{Lemma}
\newtheorem{proposition}[theorem]{Proposition}
\newcommand{\Mod}[1]{\ (\mathrm{mod}\ #1)}
\begin{document}
\title{Ramanujan $\tau$-Functions}
\author{Yu Xiong}
\maketitle
\begin{definition}
    For any $A \in GL(2,\mathbb{R})$, the \textcolor{blue}{slash operator} defined on $f:\mathcal{H}\to \mathbb{C}$ is
    \[
        f|_kA(z)=(\text{det}(A))^{k/2}j_A(z)^{-k}f(Az),
    \] we have $j_A(z)=cz+d$, if $A=\begin{pmatrix}
        a&b\\
        c&d
    \end{pmatrix}$.
\end{definition}
\begin{definition}
    \[\Delta_n:=\left\{\begin{pmatrix}
        a&b\\
        0&d
    \end{pmatrix}\big| ad=n, 0\leq b < d\right\}.\]
\end{definition}
\begin{lemma}
    There is a one to one correspondence \[\Delta_n\times SL(2,\mathbb{Z})\leftrightarrow  SL(2,\mathbb{Z}) \times \Delta_n.\] That is for any $\rho \in \Delta_n$, $\tau \in SL(2,\mathbb{Z})$, there exist unique $\tau'\in \Gamma,\rho' \in \Delta_n$, such that $\rho \tau =\tau' \rho'$.
\end{lemma}
\begin{proof}
    If $\rho=\begin{pmatrix}
        a&b\\
        0&d
    \end{pmatrix} \in \Delta_n$ and $\tau=\begin{pmatrix}
        \alpha&*\\
        \gamma&\delta
    \end{pmatrix} \in SL(2,\mathbb{Z})$, we want to find $\tau'= \begin{pmatrix}
        \alpha'&*\\
        \gamma'&\delta'
    \end{pmatrix} \in SL(2,\mathbb{Z})$ and $\rho'\begin{pmatrix}
        a'&b'\\
        0&d'
    \end{pmatrix} \in \Delta_n$ such that $\rho \tau =\tau' \rho'$. This is we want to solve:\[
    \begin{pmatrix}
        a&b\\
        0&d
    \end{pmatrix}
    \begin{pmatrix}
        \alpha&*\\
        \gamma&\delta
    \end{pmatrix}=
    \begin{pmatrix}
        \alpha'&*\\
        \gamma'&\delta'
    \end{pmatrix}
    \begin{pmatrix}
        a'&b'\\
        0&d'
    \end{pmatrix}.
    \]
    This is \[
    \begin{pmatrix}
        a\alpha+b\gamma&*\\
        d\gamma&d\delta
    \end{pmatrix}=
    \begin{pmatrix}
        a'\alpha'&*\\
        a'\gamma'&b'\gamma'+d'\delta'
    \end{pmatrix},
    \]
    and this is equal to
    \begin{equation*}
        \begin{cases}
            a'\alpha'&=a\alpha+b\gamma;\\
            a'\gamma'&=d\gamma;\\
            b'\gamma'+d'\delta'&=d\delta.
        \end{cases}
    \end{equation*}
    For $\gamma\neq 0$, We can see that $a'\mid a\alpha+b\gamma$ and $a' \mid d\gamma$, so \[(\frac{a\alpha+b\gamma}{a'},\frac{d\gamma}{a'})=(\alpha',\beta').\] We can see that $(\alpha',\beta')=1$, so $a'=(a\alpha+b\gamma,d\gamma)$. In addition, we have \begin{equation}\label{eqn 1}
        \begin{cases}
            a'&=(a\alpha+b\gamma,d\gamma);\\
            d'&=n/(a\alpha+b\gamma,d\gamma);\\
            \alpha'&=(a\alpha+b\gamma)/(a\alpha+b\gamma,d\gamma);\\
            \gamma'&=d\gamma/(a\alpha+b\gamma,d\gamma).
        \end{cases}
    \end{equation}
    % We have $\alpha'\delta' \equiv 1 \Mod{|\gamma'|}$, so $\delta'\equiv x \Mod{|\gamma'|}$, for some $x\in\mathbb{Z}$, and we have $\delta'=x+n\gamma'$. We can see that for any $n\in \mathbb{Z}$, we can make $\delta'\alpha'-*\gamma'=1$. So we have $\gamma'b'+d'(x+\gamma' n)=\delta d$, and that is \begin{equation}\label{1}
    % \gamma'(b'+nd')=\delta d-xd'.
    % \end{equation}
    % We can see that $\gamma'\neq 0$, if $\gamma\neq 0$, and if we modulo $|\gamma'|$ for RHS of equation \ref{1}, we will have \[\delta d-xd'\equiv \delta d- d'/\alpha'\equiv \frac{n\alpha\delta+b\gamma\delta d-n}{a\alpha+b\gamma}=0\Mod{|\gamma'|}.\]Hence, we have $b'=\frac{\delta d-xd'}{\gamma'}-nd'$ and taking a suitable $n$, we get the $b'$.\\
    % If $\gamma=0$, then we have 
    % \[\begin{pmatrix}
    %     a&b\\
    %     0&d
    % \end{pmatrix}\begin{pmatrix}
    %     1&u\\
    %     0&1
    % \end{pmatrix}=
    % \begin{pmatrix}
    %     1&v\\
    %     0&1
    % \end{pmatrix}
    % \begin{pmatrix}
    %     a&b'\\
    %     0&d
    % \end{pmatrix},\]
    % so we take $b'\equiv au+b\Mod{d}$ and $v=d^{-1}(au+b-dv)$.\\
    % For surjection, we need some matrices transform.
    We can see that $a',\alpha',\gamma'$ are positive integers. For $d'$, we have \[(a,\gamma)=(a\alpha,\gamma)=(a\alpha+b\gamma,\gamma),\]is a divisor of $a$ due to $(\alpha,\gamma)=1$, so we have $(a\alpha+b\gamma,d\gamma)\mid ad$ and $d$ is an integer. What we need is prove \begin{equation}\label{eqn 2}
        (\gamma',d')\mid d\delta,
    \end{equation} and we will get a unique $0 \leq b'<d'$ and $\delta'$ from Bezout's theorem. From equation \ref{eqn 1}, we have $(\gamma',d')=(ad,d\gamma)/(a\alpha+b\gamma,d\gamma)$, and proving \ref{eqn 2} is equal to prove \[(a,\gamma)\mid \delta(a\alpha+b\gamma,d\gamma).\] But this is from $(a,\gamma)\mid (a\alpha+b\gamma)$ and $(a,\gamma)\mid \gamma$.

    If $\gamma=0$, then we have 
    \[\begin{pmatrix}
        a&b\\
        0&d
    \end{pmatrix}\begin{pmatrix}
        1&u\\
        0&1
    \end{pmatrix}=
    \begin{pmatrix}
        1&v\\
        0&1
    \end{pmatrix}
    \begin{pmatrix}
        a&b'\\
        0&d
    \end{pmatrix},\]
    so we take $b'\equiv au+b\Mod{d}$ and $v=d^{-1}(au+b-dv)$, and $v$ is an integer for $d\mid (au+b-dv)$.

    For surjection, we need some matrices transform. If we have $\tau'\in \Gamma$, $\rho'\in \Delta_n$, we want to find $\tau$ and $\gamma$. We can assume that $\tau'\rho'=\begin{pmatrix}
        x&y\\
        z&w
    \end{pmatrix}$, where $xw-yz=n$. Then we have 
    \[
    \begin{pmatrix}
        x&y\\
        z&w
    \end{pmatrix}
    \begin{pmatrix}
        w/(z,w)&*\\
        -z/(z,w)&*
    \end{pmatrix}=
    \begin{pmatrix}
        *&*\\
        0&*
    \end{pmatrix}.
    \]With a matrix like $\begin{pmatrix}
        1&n\\
        0&1
    \end{pmatrix}$ we will get the surjection.
\end{proof}
\end{document}